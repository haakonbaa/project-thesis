\chapter{Modeling}

\section{Rigid Body Kinetics}

Before modeling a general underwater vehicle manipulator system with multiple links, we will model
a rigid body under water. As it turns out, the equations describing the motion
of a system with multiple links is very much related to the dynamic properties
of each individual link. This will become apparent when describing the mass
and coreolis matrices, as well as forces acting on the general system.
Consider a rigid body with mass $m$. Denote by $\bm{v}_{ng}^b$ the bodys translational
velocity, and by $\bm{\omega}_{ng}^b$ its angular velocity in the
body-fixed frame. We can find the kinetic energy of the body by integrating the
kinetic energy of every infinitesimal mass element.
\begin{equation}
    T = \frac{1}{2} \int \bm{v}_m^T \bm{v}_m\, dm,
\end{equation}
where $\bm{v}_m$ is the velocity of the mass element.
The velocity of the mass element $\bm{v}_m$ can be expressed as
\begin{align}
    \bm{v}_m &= \bm{v}_{ng}^b + \bm{\omega}_{ng}^b \times \bm{r}_m,
\end{align}
where $\bm{r}_m$ is the position of the mass element relative to the center of
mass. By defining $\bm{\nu} = \bm{\nu}_{bg}^b = \begin{bmatrix}(\bm{v}_{ng}^b)^T & (\bm{\omega}_{ng}^b)^T \end{bmatrix}^T$, the kinetic energy can be written as
\begin{align}
    T = \frac{1}{2} \bm{\nu}^T
    \begin{bmatrix}
        \I_{3\times 3} \int \,dm & - \int [\bm{r}_m]_\times \,dm \\
        - \int [\bm{r}_m]_\times \,dm & -\int [\bm{r}_m]_\times [\bm{r}_m]_\times \,dm
    \end{bmatrix}
    \bm{\nu}\label{eq:T_body}
\end{align}
Since the twist is defined in the center of gravity, we have that
\begin{align}
    \int [\bm{r}_m]_{\times} \,dm &= \bm{0},
\end{align}
by definition of the center of mass. The bottom right block in \autoref{eq:T_body} is recognized as the
definition of the inertia matrix $\bm{I_g^b}$. The kinetic energy can then be
expressed as
\begin{subequations}
\begin{align}
    \bm{M} &= \begin{bmatrix} m\I_{3} & \bm{0} \\ \bm{0} & \bm{I}_g^b \end{bmatrix} \\
        T &= \frac{1}{2} \bm{\nu}^T \bm{M} \bm{\nu}. \label{eq:TM_body}
\end{align}
\end{subequations}
The matrix $\bm{M}$ in \autoref{eq:TM_body} is called the spatial inertia matrix of the rigid body.
Using Lagrange's method, being careful when differentiating in a rotating frame,
we can find the equations of motion for the rigid body as \cite{fossen2021}:
\begin{align}
    \underbrace{
    \begin{bmatrix}
        m \I_{3} & \bm{0} \\
        \bm{0} & \bm{I}_g^b
    \end{bmatrix}
}_{\bm{M}_{RB}^{CG}} \dot{\bm{\nu}}
    +
    \underbrace{
    \begin{bmatrix}
        m [\bm{\omega}_{nb}^b]_{\times} & \bm{0} \\
        \bm{0} & -[\bm{I}_g^b \bm{\omega}_{nb}^b]_{\times}
    \end{bmatrix}
}_{\bm{C}_{RB}^{CG}}\bm{\nu} = \bm{\tau}
    \label{eq:rigid_body_eom_cg}
\end{align}
where $\bm{\tau}$ is the generalized forces acting on the body. We call the matrices
pre-multiplying $\dot{\bm{\nu}}$ and $\bm{\nu}$ for $\bm{M}_{RB}^{CG}$ and $\bm{C}_{RB}^{CG}$ respectively. In general one might want to consider the equations of motion
about a different point than the center of gravity. The twist of the body is
related to the twist in the new point by the matrix \cite{fossen2021}

\begin{align}
        \bm{\nu}_{ng}^{b} =
    \underbrace{
        \begin{bmatrix}
            \I_{3} & -[\bm{r}_{ng}^{b}]_{\times} \\
            \bm{0} & \I_{3}
        \end{bmatrix}
    }_{\bm{H}(\bm{r}_{ng}^b)}
    \bm{\nu}_{nb}^{b},
    \label{eq:fossen_H}
\end{align}

where $\bm{r}_{ng}^{b}$ is the position of the center of gravity relative to the
new point and $\bm{\nu}_{nb}^{b}$ is the twist in the new point. By
pre-multiplying \autoref{eq:rigid_body_eom_cg} by the inverse of the matrix
in \autoref{eq:fossen_H} we get the equations of motion about the new point:
\begin{align}
    \underbrace{
        \bm{H}^T(\bm{r}_{bg}^b) \bm{M}_{CG}^{RB} \bm{H}(\bm{r}_{bg}^b)
    }_{\bm{M}_{RB}}
    \dot{\bm{\nu}}_{nb}^b
    + \underbrace{
        \bm{H}^T(\bm{r}_{bg}^b) \bm{C}_{CG}^{RB} \bm{H}(\bm{r}_{bg}^b)
    }_{\bm{C}_{RB}}
    \bm{\nu}_{nb}^b
    = \underbrace{
        \bm{H}^T(\bm{r}_{bg}^b)
    \bm{\tau}
    }_{\bm{\tau}_{b}^b}.
\end{align}
$\bm{\tau}_{b}^b$ is the generalized forces acting on the body about the new
point. The $\bm{M}_{RB}$ and $\bm{C}_{RB}$ matrices are given by \cite{fossen2021}:
\begin{subequations}
\begin{align}
    \bm{M}_{RB} &= \begin{bmatrix}
        m \I_3 & - m [\bm{r}_{bg}^b]_{\times} \\
        m [\bm{r}_{bg}^b]_{\times} & \bm{I}_g^b - m [\bm{r}_{bg}^b]_{\times}[\bm{r}_{bg}^b]_{\times}
    \end{bmatrix} \\
    \bm{C}_{RB} &= \begin{bmatrix}
        m [\bm{\omega}_{nb}^b] & -m [\bm{\omega}_{nb}^b]_{\times}[\bm{r}_{bg}^b]_{\times} \\
        m [\bm{\omega}_{nb}^b]_{\times}[\bm{r}_{bg}^b]_{\times} & 
        -m \cross{\bm{r}_{bg}^b} \cross{\bm{\omega}_{bg}^b} \cross{\bm{r}_{bg}^b} -
        \cross{\bm{I}_g^b \bm{\omega}_{bg}^b}
    \end{bmatrix}. \label{eq:crb}
\end{align}
\end{subequations}
It can be shown that the coreolis matrix in \autoref{eq:crb} ca be equivallenty
written as \cite{sagatun1991}:
\begin{align}
    \bm{C}_{RB}(\bm\nu) = \begin{bmatrix}
        \bm{0}_{3 \times 3} & 
        -\left[\bm{M}_{11}\bm{v}_{nb}^{b} + \bm{M}_{12}\bm{\omega}_{nb}^b \right]_{\times} \\
        -\left[\bm{M}_{11}\bm{v}_{nb}^{b} + \bm{M}_{12}\bm{\omega}_{nb}^b \right]_{\times} &
        -\left[\bm{M}_{21}\bm{v}_{nb}^{b} + \bm{M}_{22}\bm{\omega}_{nb}^b \right]_{\times}
    \end{bmatrix},
\end{align}
Where $\bm{M}_{ij}$ are the $3 \times 3$ submatrices of the $\bm{M}_{RB}$ matrix.

% -----------------------------------------------------------------------------
\newpage
\section{AIAUV Modeling}

A comprehensive description of the kinematics and dynamics of articulated
underwater robots can be found in \cite{schmidt2018}. Much of this section
will be based on the work of \cite{schmidt2018}, as well as an unpublished
first draft of this paper. The following chapter will use notation adopted from
\cite{from2014}.

\subsection{Forward Kinematics}
\label{sec:forward_kinematics}
Consider a system with $n$ rigid links connected together with $n-1$ joints
of one parameter. The links are indexed from $1$ to $n$, where the base
has index $1$ and the tail has index $n$. Let
\begin{align}
    \bm{H}_i = \begin{bmatrix}
        \bm{R}_i^n & \bm{p}_{ni}^n \\
        \bm{0}^T & 1
    \end{bmatrix} \in \SE,
\end{align}
be the homogeneous transformation matrix from frame $i$, attached to link $i$,
to the inertial NED\footnote{Atleast approximately inertial} frame $n$. Define
\begin{align}
    \bm{A}_i(\theta_i) &\in \SE & i &= 1, 2, \ldots, n-1
\end{align}
as the transformation matrix from frame $i+1$ to frame $i$ such that
\begin{align}
    \bm{H}_{i+1} &= \bm{H}_i \bm{A}_i(\theta_i).
\end{align}
Expanding this recursive relation gives
\begin{align}
    \bm{H}_i &= \bm{H}_1 \bm{A}_1(\theta_1) \bm{A}_2(\theta_2) \cdots \bm{A}_{i-1}(\theta_{i-1}).
\end{align}
We define $\bm{\theta} = \begin{bmatrix}\theta_1 & \theta_2 & \cdots & \theta_{n-1}\end{bmatrix}^T$
and
\begin{align}
    \bm{A}_{i:j}(\bm{\theta}) =
    \begin{cases}
        \bm{A}_i(\theta_i) \bm{A}_{i+1}(\theta_{i+1}) \cdots \bm{A}_{j-1}(\theta_{j-1}) & i \leq j \\
        \bm{0}_{4 \times 4} & i > j \\
    \end{cases}
\end{align}
It can be shown that the transformation matrix $\bm{A}_{i}(\theta_i)$ can be
expressed as \cite{murray2017}:
\begin{align}
    \bm{A}_i(\theta_i) &= \bm{A}_i(0) \exp([\bm{a}_i]_{\wedge} \theta_i), \label{eq:expmap}
\end{align}
for some $\bm{a}_i \in \R^6$. Combining \autoref{eq:expmap} with \autoref{eq:body_twist_def}, the body twist
of link $i+1$ relative to the inertial frame $n$ can be expressed as
\begin{subequations}
\begin{align}
    \bm{\nu}_{n(i+1)}^{i+1} &= \left(\bm{H}_{i+1}^n\right)^{-1} \dot{\bm{H}}_{i+1}^n \\
    &= \bm{A}_{1:i}^{-1}(\bm\theta) \bm{H}_1^{-1} \dot{\bm{H}}_{1} \bm{A}_{1:i}(\bm\theta) \nonumber \\
    &+ \bm{A}_{2:i}^{-1}(\bm\theta) [\bm{a}_1]_{\wedge}\bm{A}_{2:i}(\bm\theta) \dot{\theta}_1 \nonumber \\
    &\vdots \label{eq:body_twist_def} \\
    &+ \bm{A}_{i:i}^{-1}(\bm\theta) [\bm{a}_{i-1}]_{\wedge}\bm{A}_{i:i}(\bm\theta) \dot{\theta}_{i} \nonumber \\
    &+ [\bm{a}_{i}]_{\wedge} \dot{\theta}_i. \nonumber
\end{align}
\end{subequations}
Defining $\bm{\nu}_i := \bm{\nu}_{ni}^i$ as the body twist of link $i$ relative to
the inertial frame $n$, \autoref{eq:body_twist_def} can be written recursively as:
\begin{subequations}
    \label{eq:body_twist_recursive}
\begin{align}
    \bm{\nu}_{1} &= \bm{\nu}_{n1}^1 \\
    \bm{\nu}_{i+1} &= \Ad^{-1}(\bm{A}_i(\theta_i)) \bm{\nu}_i + [\bm{a}_i]_{\wedge} \dot{\theta}_i.
\end{align}
\end{subequations}
This motivates the definition of the link Jacobians that are defined in a way
such that
\begin{align}
    \bm{\nu}_{i} = \bm{J}_i(\bm{\theta}) \bm{\zeta}
\end{align}
where the body twist $\bm{\nu}$ and the joint velocities $\dot{\bm{\theta}}$ are
collected in the vector $\bm{\zeta} \in \R^{6+(n-1)}$
\begin{align}
    \bm{\zeta} = \begin{bmatrix}\bm{\nu} \\ \dot{\bm{\theta}}\end{bmatrix}.
\end{align}
From \autoref{eq:body_twist_recursive} we can see that the link Jacobians can be
defined as
\begin{subequations}
    \label{eq:link_jacobian}
\begin{align}
    \bm{J}_1(\bm{\theta}) &= \begin{bmatrix} \bm{I}_6 & \bm{0}_{6 \times n} \end{bmatrix} \\
        \bm{J}_{i+1}(\bm{\theta}) &= \begin{bmatrix}
            \Ad^{-1}(\bm{A}_{1:i}(\bm{\theta}))  &  \Ad^{-1}(\bm{A}_{2:i}(\bm{\theta})) \bm{a}_1 &
            \cdots & \bm{a}_i & \bm{0}_{6 \times (n-i)}
        \end{bmatrix} \\
    &= \Ad^{-1}(\bm{A}_i(\theta_i)) \bm{J}_i(\bm{\theta}) + \begin{bmatrix}
        \bm{0}_{6 \times (5+i)} & \bm{a}_i & \bm{0}_{6 \times (n-i)}
    \end{bmatrix}.
\end{align}
\end{subequations}
The derivatives of the link jacobians are needed for control purposes and
calculation of the coreolis matrix. By differentiating \autoref{eq:link_jacobian}
with respect to time, we get
\begin{subequations}
\begin{align}
    \dot{\bm{J}}_1(\bm{\theta}) &= \bm{0}_{6\times(6+n)} \\
    \dot{\bm{J}}_{i+1}(\bm{\theta},\dot{\bm{\theta}}) &= -[\bm{a}_i]_{\wedge} \Ad^{-1}(\bm{A}_i(\theta_i))
        \bm{J}_i(\bm{\theta})\dot{\theta}_i + \Ad^{-1}(\bm{A}_i(\theta_i))\dot{\bm{J}}_i(\bm{\theta},\dot{\bm{\theta}}) \\
    &= -[\bm{a}_i]_{\wedge}\bm{J}_{i+1}(\bm{\theta})\dot{\theta}_i +
        \Ad^{-1}(\bm{A}_i(\theta_i))\dot{\bm{J}}_i(\bm{\theta},\dot{\bm{\theta}}).
\end{align}
\end{subequations}

\subsection{Differential Kinematics}
Consider the same system as described in the introduction of \autoref{sec:forward_kinematics}.
The configuration of the entire system is described by the vector $\bm{\xi} \in \R^{5+n}$
\begin{align}
    \bm{\xi} &= \begin{bmatrix}\bm{\eta} \\ \bm{\theta} \end{bmatrix} &
        \bm{\eta} &= \begin{bmatrix}\bm{p}_{nb}^n \\ \bm{\Theta}_{nb} \end{bmatrix}
\end{align}
Velcity
\begin{align}
    \bm{\zeta} &= \begin{bmatrix}\bm{\nu}_{nb}^b \\ \dot{\bm{\Theta}}\end{bmatrix} &
        \bm{\nu}_{nb}^b = \begin{bmatrix} \bm{v}_{nb}^b \\ \bm{\omega}_{nb}^b\end{bmatrix}
\end{align}
\begin{subequations}
\begin{align}
    \dot{\bm{\xi}} &= \bm{J}_{\xi}(\bm{\Theta}_{nb})\bm{\zeta} \\
    \bm{J}_{\xi}(\bm{\Theta}_{nb}) &= \begin{bmatrix}
        \bm{R}_{nb}(\bm{\Theta}_{nb}) & \bm{0}_{3 \times 3} & \bm{0}_{3 \times n} \\
        \bm{0}_{3 \times 3} & \bm{T}_{nb}(\bm{\Theta}_{nb}) & \bm{0}_{3 \times n} \\
        \bm{0}_{n \times 3} & \bm{0}_{n \times 3} & \I_{n}
    \end{bmatrix}
\end{align}
\end{subequations}
\cite{fossen2021}:
\begin{align}
    \bm{T}_{nb}(\bm{\Theta}_{nb}) = \begin{bmatrix}
        1 & \sin\phi \tan\theta & \cos \phi \tan \theta \\
        0 & \cos \phi & -\sin\phi \\
        0 & \sin \phi / \cos \theta & \cos \phi / \cos \theta
    \end{bmatrix}
\end{align}

\subsection{Dynamics}

For each link, $i$, defined in the previoud subchapter, we associate the following
properties: $\bm{M}_i \in \R^{6 \times 6}$ is the inertia matrix of the link,
$\bm{J}_i \in \R^{6 \times (6+n)}$ is the link jacobian, $\bm{C}_i \in \R^{6 \times 6}$
is the coreolis matrix, $\bm{D}_i \in \R^{6 \times 6}$ is the damping matrix and
$\bm{g}_i \in \R^{6}$ is the gravitational forces and buoyancy forces and moments.
The damping matrix will be discussed in \autoref{sec:hydrodynamics}.
$\bm{M}_i$ takes into account the added mass of the link
\begin{align}
    \bm{M}_i &= \bm{M}_{RB,i} + \bm{M}_{A,i},
\end{align}
where $\bm{M}_{RB,i}$ is the rigid body inertia matrix of the link and $\bm{M}_{A,i}$
is the added mass matrix of the link. As a result of the added mass, the coreolis
matrix is given by
\begin{align}
    \bm{C}_i &= \bm{C}_{RB,i} + \bm{C}_{A,i}
\end{align}
where $\bm{C}_{RB,i}$ is the coreolis matrix of the rigid body and $\bm{C}_{A,i}$
is a function of the added mass matrix. The added mass matrix and the damping matrix
for each link is the topic of \autoref{sec:hydrodynamics}. The dynamics of the
system described in the previous subchapter can be modeled as
\cite{from2014}:
\begin{align}
    \bm{M}(\bm{\theta})\dot{\bm{\zeta}} +
        \bm{C}(\bm{\theta}, \bm{\zeta}) \bm{\zeta} +
        \bm{D}(\bm{\theta}, \bm{\zeta}) \bm{\zeta} +
        \bm{g}(\bm{\xi}) =
        \bm{\tau} + \bm{\tau}_{\mathrm{ext}}.
\end{align}
$\bm{M}(\bm{\theta}) \in \R^{(6+n)\times(6+n)}$  is the inertia matrix of the
system, $\bm{C}(\bm{\theta}, \bm{\zeta}) \in \R^{(6+n)\times(6+n)}$ is the
coreolis matrix, $\bm{D}(\bm{\theta}, \bm{\zeta}) \in \R^{(6+n)\times(6+n)}$ is
the damping matrix, $\bm{g}(\bm{\xi}) \in \R^{6+n}$ is the gravitational forces
and buoyancy forces and moments acting on the system, $\bm{\tau} \in \R^{n}$ is the generalized
forces acting on the system as a result of the actuators,
and $\bm{\tau}_{\mathrm{ext}} \in \R^{6+n}$ is the
generalized external forces acting on the system. Let $\bm{J}_{i}(\bm{\theta})$
be the link jacobians as defined in \autoref{eq:link_jacobian}.
The inertia matrix
can be expressed as
\begin{align}
    \bm{M}(\bm{\theta}) &= \sum_{i=1}^{n} \bm{J}_{i}^T(\bm{\theta}) \bm{M}_i \bm{J}_{i}(\bm{\theta}).
\end{align}
Note that if the jacobians are full rank, the inertia matrix is positive definite.
Furthermore, it is allways symmetric. The coreolis matrix can be expressed as
\begin{align}
    \bm{C}(\bm{\theta}, \bm{\zeta}) &=
    \sum_{i=1}^{n} \bm{J}_{i}^T(\bm{\theta}) \bm{M}_i \dot{\bm{J}}_{i}(\bm{\theta},\dot{\bm{\theta}})
    -\bm{J}_{i}^T(\bm{\theta}) \bm{C}_i(\bm{\theta},\bm{\zeta})_i \bm{J}_{i}(\bm{\theta}) \bm{\zeta}.
\end{align}
The damping matrix can be expressed as
\begin{align}
    \bm{D}(\bm{\theta}, \bm{\zeta}) &=
    \sum_{i=1}^{n} \bm{J}_{i}^T(\bm{\theta}) \bm{D}_i(\bm{\theta},\bm{\zeta}) \bm{J}_{i}(\bm{\theta}).
\end{align}
and the gravitational forces and buoyancy forces and moments can be expressed as
\begin{align}
    \bm{g}(\bm{\xi}) &=
    \sum_{i=1}^{n} \bm{J}_{i}^T(\bm{\theta}) \bm{g}_i(\bm{\xi}).
\end{align}
The control inputs $\bm{u}$ are mapped to the generalized forces $\bm{\tau}$ by
the actuator configuration matrix $\bm{B}(\bm{\theta})$ as
\begin{align}
    \bm{\tau} &= \bm{B}(\bm{\theta}) \bm{u} &
    \bm{B}(\bm{\theta}) &= \begin{bmatrix}
        \bm{J}_1^T(\bm{\theta}) \bm{B}_1 & \cdots & \bm{J}_n^T(\bm{\theta}) \bm{B}_n
    \end{bmatrix},
\end{align}
where $\bm{B}_i$ is the actuator configuration matrix for link $i$. For a simple
thruster configuration, such as in the case where all thrusters are fixed with
respect to some link, the $\bm{B}_i$ matrices are constant and can be expressed
as
\begin{align}
    \bm{B}_i &= \begin{bmatrix}
        \bm{\beta}_{t,i,1} & \cdots & \bm{\beta}_{t,i,m} \\
        \bm{r}_{t,i,1} \times \bm{\beta}_{t,i,1} & \cdots & \bm{r}_{t,i,m} \times \bm{\beta}_{t,i,m}
    \end{bmatrix},
\end{align}
where $\bm{\beta}_{t,i,j}$ is the direction of the thrust and $\bm{r}_{t,i,j}$
is the point of application of the thrust.





% -----------------------------------------------------------------------------
\section{Hydrodynamics}
\label{sec:hydrodynamics}

Hydrodynamics is the study of the forces acting on a body in a fluid. The forces
can be modeled as potential forces, giving rise to the added mass matrix, and
damping forces. This chapter will focus on how to model the hydrodynamic forces
acting on a rigid body in a fluid.

\subsection{Added Mass}

When a rigid body accelerates in a fluid, the fluid is accelerated as well. 
This contributes to the total kinetic energy of the system \cite{antonelli2018}.
To compensate for these effects, the spatial inertia matrix of the rigid body
can be augmented to approximate the total kinetic energy of the system: 
\begin{equation}
    \bm{M} = \bm{M}_{RB} + \bm{M}_{A}.
\end{equation}
The matrix $\bm{M}_A$ is in reality a function of the wave excitation frequency
of the waves in the ocean \cite{fossen2021}. Because of the complex computations
of $\bm{M}_A(\omega)$, as well as the fact that for large vessels the natural
frequencies of the vessel are much lower than the wave excitation frequencies,
the added mass matrix is approximated as the zero-frequency added mass matrix
\begin{align}
    \bm{M}_A(\omega) &\approx \bm{M}_A(0)
\end{align}
Under this assumption, the kinetic energy of the fluid $T_A$ is approximated
as
\begin{align}
    T_a &= \frac{1}{2}\bm{\nu}^T\bm{M}_A\bm{\nu} & \dot{\bm{M}_A} &= \bm{0},
\end{align}
where $\bm{\nu}$ is the twist velocity of the rigid body. For underwater 
vehicles at low speed where the the shape has three planes of symmetry, the
added mass matrix can be approximated as \cite{fossen2021}:
\begin{align}
    \bm{M}_A = \bm{M}_A^T =
    -\operatorname{diag}(X_{\dot{u}}, Y_{\dot{v}}, Z_{\dot{w}},
        K_{\dot{p}}, M_{\dot{q}}, N_{\dot{r}}).
\end{align}
According to \cite{fossen2021} this approximation is found to be quite good for
many applications due to the fact that the off diagonal coupling terms are small
compared to the diagonal terms.

By applying strip theory the added mass matrix for a cylindirical rigid body of
mass $m$, lenght along the x-axis $l$ and radius $r$ can be derived as \cite{fossen1994}:
\begin{align}
 X_{\dot{u}} &= -0.1 m &
 Y_{\dot{v}} &= -\pi \rho r^2 l \nonumber \\
 Z_{\dot{w}} &= -\pi \rho r^2 l &
 K_{\dot{p}} &= 0 \\
 M_{\dot{q}} &= -\frac{1}{12} \pi \rho r^2 l^3 &
 N_{\dot{r}} &= -\frac{1}{12} \pi \rho r^2 l^3 \nonumber
\end{align}
where $\rho$ is the density of the fluid. For a three-dimensional ellipsoid with
lengths $a$, $b$ and $c$ along the x, y and z axis respectively, the added mass
can be approximated as \cite{fossen2021}:
\begin{align}
 X_{\dot{u}} &= -\frac{\alpha_0}{2-\alpha_0}m &
 Y_{\dot{v}} &= -\frac{\beta_0}{2-\beta_0}m \nonumber \\
 Z_{\dot{w}} &= Y_{\dot{v}}&
 K_{\dot{p}} &= 0 \\
 M_{\dot{q}} &= -\frac{1}{5}\frac{(b^2-a^2)^2(\alpha_0-\beta_0)}{2(b^2-a^2) + (b^2+a^2)(\beta_0-\alpha_0)}m&
 N_{\dot{r}} &= M_{\dot{q}} \nonumber
\end{align}

\subsection{Damping}

There are several phenomena that contribute to the damping of a rigid body in
a fluid. Among these are potential damping, skin friction, wave drift damping,
damping due to vorex shedding and lifting forces. In many cases
all of these effects can be approximated using a linear damping matrix $\bm{D}$ and
a quadratic damping matrix $\bm{D}_n$ \cite{fossen2021}. Damping can be modeled
as
\begin{align}
    \bm{D}(\bm{\nu}_r) &= \bm{D} + \bm{D}_n(\bm{\nu}_r) \\
    \bm{\tau}_d &= \bm{D}(\bm{\nu}_r)\bm{\nu}_r
\end{align}
for a 6-DOF rigid body, $\bm{D}(\bm{\nu}_r)$ is a $6\times 6$ matrix, $\bm{\nu}_r$
is the twist of the rigid body relative to the fluid and $\bm{\tau}_d \in \R^6$
is a vector of the damping forces and moments acting on the rigid body. \cite{antonelli2018}
states that the damping matrix can be approximated as
\begin{subequations}
\begin{align}
    \bm{D} &= -diag(X_u, Y_v, Z_w, K_p, M_q, N_r) \\
    \bm{D}_n &= -diag(X_{u|u|}|u|, Y_{v|v|}|v|, Z_{w|w|}|w|, K_{p|p|}|p|, M_{q|q|}|q|, N_{r|r|}|r|)
\end{align}
\end{subequations}

\citeauthor{mcmillan1995} shows that by using strip theory, the damping force and
moment on a cylinder can be approximated by the follwing integrals \cite{mcmillan1995}:
\begin{subequations}
    \label{eq:cylinder_damping}
    \begin{align}
        \bm{f}_d &= - \rho C_D r \int_{0}^{l} ||\bm{v}^n(x)|| \bm{v}^n(x) \,dx \\
        \bm{m}_d &= - \rho C_D r \int_{0}^{l} ||\bm{v}^n(x)||
        \left(\begin{bmatrix}x & 0 & 0\end{bmatrix}^T \times \bm{v}^n(x)\right) \,dx,
    \end{align}
\end{subequations}
where $r$ and $l$ are the radius and length of the cylinder, $\rho$ the density
of the fluid, $C_D$ the drag coefficient and $\bm{v}^n(x)$ the velocity of the
fluid at the point $x$ along the length of the cylinder. Using the results stated
in \autoref{eq:cylinder_damping}, \citeauthor{schmidt2018} proposes a linear
damping matrix for a cylinder as \cite{schmidt2018}:
\begin{align}
    \bm{D} = \rho \pi l C_D v_{ref}
    \begin{bmatrix}
        \beta &            0 &             0 &          0 &              0 &            0 \\
            0 &            1 &             0 &          0 &              0 & \frac{1}{2}l \\
            0 &            0 &             1 &          0 &  -\frac{1}{2}l &            0 \\
            0 &            0 &             0 & \gamma r^2 &              0 &            0 \\
            0 &            0 & -\frac{1}{2}l &          0 & \frac{1}{3}l^2 &            0 \\
            0 & \frac{1}{2}l &             0 &          0 &              0 & \frac{1}{3}l^2
    \end{bmatrix},
\end{align}
where $C_D$, $v_{ref}$, $\beta$, $\gamma$ are constants to be determined.

% -----------------------------------------------------------------------------
\iffalse
\section{UMS Kinematics}

The following chapter will descibe the kinematics of an underwater vehicle
maipulator system (UVMS). The UVMS consists of a rigid base and is connected to
a manipulator with $n$ links. This makes the UVMS have a snake-like structure.
Adopting notation from \cite{fossen2021}, the rigid base can be uniquely described
by a set of $6$ coordinates
\begin{align}
    \bm{\eta} &= \begin{bmatrix} \bm{p} \\ \bm{\Theta} \end{bmatrix} \in \R^6 &
        \bm{p} &= \begin{bmatrix} x^n \\ y^n \\ z^n \end{bmatrix} \in \R^3 &
    \bm{\Theta} &= \begin{bmatrix} \phi \\ \theta \\ \psi \end{bmatrix} \in \R^3,
\end{align}
where $\bm{p}$ is the position of the base described in a NED frame and $\bm{\Theta}$
are the Euler angles describing the orientation of the base. Note that the Euler
angles are singular at $\theta = \pm \pi/2$, and that using quaternions
would avoid this problem. Introducing quaternions would however add an extra
equality constraint to the system, making the system harder to simulate. Because
of this, Euler angles are used in this thesis. 

We assume the position of the manipulator links relative to the base are uniquely
described by a set of $n$


% -----------------------------------------------------------------------------
\section{UMS Dynamics}



% mention that the damping matrix is a function of the velocity of the body
% mention that for UVMs the damping matrix can be assumed decoupled for all links.

{
    \color{red}
    \begin{itemize}
        \item Multi-body dynamics
        \item Hydrodynamics (damping)
        \item Jacobians
        \item Inspiration from Henrik?
    \end{itemize}
}
\fi
