Light-UVMS, including 
ROVs and articulated intervention-AUVs, 
are gaining prominence in subsea industries such as harbor control and 
environmental monitoring. These systems present unique control challenges due 
to the strong coupling between base and manipulator arm motions, a phenomenon 
less pronounced in heavier work-class ROVs. This thesis explores task-priority 
control methodologies to address these challenges at both kinematic and dynamic 
levels. A comprehensive simulation framework is developed to facilitate the 
implementation and comparison of these control methods, minimizing the gap 
between simulation and real-world performance. Specifically, kinematic-level 
task-priority control focuses on velocity and acceleration redundancy 
resolution, while dynamic-level control incorporates feedback linearization and 
dynamically consistent inverses to ensure robust task execution 
despite modeling inaccuracies and disturbances. The simulation results 
highlight the strengths  and trade-offs of both control approaches, providing 
insights into their performance, computational efficiency, and robustness under 
varying operational conditions. This work lays a foundation for future 
experimental validation on NTNU’s Eelume 500 articulated intervention-AUV 
advancing the state-of-the-art in underwater robotics.
