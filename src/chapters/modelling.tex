\chapter{Modeling}

\section{Rigid Body Kinetics}

Consider a rigid body with mass $m$. Denote by $\bm{v}_{ng}^b$ the bodys translational
velocity, and by $\bm{\omega}_{ng}^b$ its angular velocity in the
body-fixed frame. We can find the kinetic energy of the body by integrating the
kinetic energy of every infinitesimal mass element.
\begin{equation}
    T = \frac{1}{2} \int \bm{v}_m^T \bm{v}_m\, dm,
\end{equation}
where $\bm{v}_m$ is the velocity of the mass element.
The velocity of the mass element $\bm{v}_m$ can be expressed as
\begin{align}
    \bm{v}_m &= \bm{v}_{ng}^b + \bm{\omega}_{ng}^b \times \bm{r}_m,
\end{align}
where $\bm{r}_m$ is the position of the mass element relative to the center of
mass. By defining $\bm{\nu} = \begin{pmatrix}(\bm{v}_{ng}^b)^T & (\bm{\omega}_{ng}^b)^T \end{pmatrix}^T$, the kinetic energy can be written as
\begin{align}
    T = \frac{1}{2} \bm{\nu}^T
    \begin{pmatrix}
        \bm{I}_{3\times 3} \int \,dm & - \int [\bm{r}_m]_\times \,dm \\
        - \int [\bm{r}_m]_\times \,dm & -\int [\bm{r}_m]_\times [\bm{r}_m]_\times \,dm
    \end{pmatrix}
    \bm{\nu}\label{eq:T_body}
\end{align}
Since the twist is defined in the center of gravity, we have that
\begin{align}
    \int [\bm{r}_m]_{\times} \,dm &= \bm{0},
\end{align}
by definition of the center of mass. The bottom right block in \autoref{eq:T_body} is recognized as the
definition of the inertia matrix $\bm{I_g^b}$. The kinetic energy can then be
expressed as
\begin{subequations}
\begin{align}
    \bm{M} &= \begin{pmatrix} m\bm{I}_{3\times 3} & \bm{0} \\ \bm{0} & \bm{I}_g^b \end{pmatrix} \\
        T &= \frac{1}{2} \bm{\nu}^T \bm{M} \bm{\nu}. \label{eq:TM_body}
\end{align}
\end{subequations}
The matrix $\bm{M}$ in \autoref{eq:TM_body} is called the spatial inertia matrix of the rigid body.
Using Lagrange's method, being careful when differentiating in a rotating frame,
we can find the equations of motion for the rigid body as \cite{fossen2021}:
\begin{align}
    \underbrace{
    \begin{pmatrix}
        m \bm{I}_{3\times 3} & \bm{0} \\
        \bm{0} & \bm{I}_g^b
    \end{pmatrix}
}_{\bm{M}_{RB}^{CG}} \dot{\bm{\nu}}
    +
    \underbrace{
    \begin{pmatrix}
        m [\bm{\omega}_{nb}^b]_{\times} & \bm{0} \\
        \bm{0} & -[\bm{I}_g^b \bm{\omega}_{nb}^b]_{\times}
    \end{pmatrix}
}_{\bm{C}_{RB}^{CG}}\bm{\nu} = \bm{\tau}
\end{align}
where $\bm{\tau}$ is the generalized forces acting on the body. We call the matrices
pre-multiplying $\dot{\bm{\nu}}$ and $\bm{\nu}$ for $\bm{M}_{RB}^{CG}$ and $\bm{C}_{RB}^{CG}$ respectively.


% -----------------------------------------------------------------------------
\subsection{Hydrodynamics}

When a rigid body accelerates in a fluid, the fluid is accelerated as well. 
This contributes to the total kinetic energy of the system \cite{antonelli2018}.
To compensate for these effects, the spatial inertia matrix of the rigid body
can be augmented to approximate the total kinetic energy of the system: 
\begin{equation}
    \bm{M} = \bm{M}_{RB} + \bm{M}_{A}.
\end{equation}
The matrix $\bm{M}_A$ is in reality a function of the wave excitation frequency
of the waves in the ocean \cite{fossen2021}. Because of the complex computations
of $\bm{M}_A(\omega)$, as well as the fact that for large vessels the natural
frequencies of the vessel are much lower than the wave excitation frequencies,
the added mass matrix is approximated as the zero-frequency added mass matrix
\begin{align}
    \bm{M}_A(\omega) &\approx \bm{M}_A(0)
\end{align}
Under this assumption, the fluikinetic energy of the fluid $T_A$ is approximated
as
\begin{align}
    T_a &= \frac{1}{2}\bm{\nu}^T\bm{M}_A\bm{\nu} & \dot{\bm{M}_A} &= \bm{0}
\end{align}
where $\bm{\nu}$ is the twist velocity of the rigid body. For underwater 
vehicles at low speed where the the shape has three planes of symmetry, the
added mass matrix can be approximated as \cite{fossen2021}:
\begin{align}
    \bm{M}_A = \bm{M}_A^T =
    \operatorname{diag}(X_{\dot{u}}, Y_{\dot{v}}, Z_{\dot{w}},
        K_{\dot{p}}, M_{\dot{q}}, N_{\dot{r}})
\end{align}
According to \cite{fossen2021} this approximation is found to be quite good for
many applications due to the fact that the off diagonal coupling terms are small
compared to the diagonal terms.

By applying srip theory the added mass matrix for a cylindirical rigid body of
mass $m$, lenght along the x-axis $l$ and radius $r$ can be derived as \cite{antonelli2018}:
\begin{align}
 X_{\dot{u}} &= -0.1 m &
 Y_{\dot{v}} &= -\pi \rho r^2 l \nonumber \\
 Z_{\dot{w}} &= -\pi \rho r^2 l &
 K_{\dot{p}} &= 0 \\
 M_{\dot{q}} &= -\frac{1}{12} \pi \rho r^2 l^3 &
 N_{\dot{r}} &= -\frac{1}{12} \pi \rho r^2 l^3 \nonumber
\end{align}
where $\rho$ is the density of the fluid. For a three-dimensional ellipsoid with
lengths $a$, $b$ and $c$ along the x, y and z axis respectively, the added mass
can be approximated as \cite{fossen2021}:
\begin{align}
 X_{\dot{u}} &= -\frac{\alpha_0}{2-\alpha_0}m &
 Y_{\dot{v}} &= -\frac{\beta_0}{2-\beta_0}m \nonumber \\
 Z_{\dot{w}} &= Y_{\dot{v}}&
 K_{\dot{p}} &= 0 \\
 M_{\dot{q}} &= -\frac{1}{5}\frac{(b^2-a^2)^2(\alpha_0-\beta_0)}{2(b^2-a^2) + (b^2+a^2)(\beta_0-\alpha_0)}m&
 N_{\dot{r}} &= M_{\dot{q}} \nonumber
\end{align}

{
    \color{red}
    \begin{itemize}
        \item Multi-body dynamics
        \item Hydrodynamics (damping)
        \item Jacobians
        \item Inspiration from Henrik?
    \end{itemize}
}
