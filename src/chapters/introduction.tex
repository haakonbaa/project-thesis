This chapter intruduces the core motivation behind the thesis and outlines the
context and importance of the work. The driving factors and key challenges that
led to the thesis are presented. A brief overview of the litterature that is
relevant to the thesis is given, and the assumptions made are
stated. The chapter concludes with a summary of the contributions of this thesis,
as well as an outline of the subsequent chapters.


% ------------------------------------------------------------------ Motivation
\section{Motivation}

% ---------------------------------------------------------- Litterature Review
\section{Literature Review}

% ----------------------------------------------------------------- Assumptions
\section{Assumptions}
\begin{enumerate}
    \item \emph{Assumption 1}
    \item \emph{Assumption 2}
    \item \emph{Assumption 3}
\end{enumerate}

% --------------------------------------------------------------- Contributions
\section{Contributions}

The main contributions as presented in this thesis are as follows:
\begin{itemize}
    \item \emph{}
    \item \emph{A Python library for mathematically modelling underwater vehicles.
    Allowing for rapid prototyping of models and fast simulations in C++.}
    \item \emph{A simulation study comparing velocity and acceleration level
        task-priority control methods for light-UVMS.}
\end{itemize}

% -------------------------------------------------------------- Thesis Outline
\section{Thesis Outline}

\emph{Chapter 2} gives an overview of the background and preliminaries of the
thesis. The chapter introduces the mathematical tools and concepts used throughout
the thesis. \emph{Chapter 3} presents the mathematical modelling of underwater
vehicle manipulator systems (UVMs). \emph{Chapter 4} introduces the Python library
\pymuvs and explains how it can be used to model underwater vehicles. \emph{Chapter 5}
presents the simulation study comparing velocity and acceleration level task-priority
control methods for light-UVMS. \emph{Chapter 6} concludes the thesis and outlines
possible future work.
