This chapter introduces the core motivation behind the thesis and outlines the
context and importance of the work. The driving factors and key challenges that
led to the thesis are presented. A brief overview of the literature that is
relevant to the thesis is given, and the assumptions made are
stated. The chapter concludes with a summary of the contributions of this thesis,
as well as an outline of the subsequent chapters.


% ------------------------------------------------------------------ Motivation
\section{Motivation}

{
    \color{red}
    \begin{itemize}
        \item Highly redundant systems $\rightarrow$ Task-priority control
        \item Inaccurate models? How good is TPC for light-UVMS?
            \begin{itemize}
                \item Kinetic vs Dynamic redundancy resolution
            \end{itemize}
        \item Need for realistic simulations and dynamic models when comparing TPC methods
    \end{itemize}
}

In 2015, the United Nations adopted the 2030 Agenda for Sustainable Development
\cite{UN2030Agenda}, which outlines 17 Sustainable Development Goals (SDGs) to
address global challenges such as poverty, inequality, climate change, and
environmental degradation. The research and methods presented in this thesis
contribute to the achievement of several of these goals. For example, \emph{SDG 14
Life Below Water}, "Conserve and sustainably use the oceans, seas and marine
resources for sustainable development.". UVMSs can monitor and inspect underwater
infrastructure, reducing the risk associated with oil spills, structural failures,
or leaks. Furthermore, UVMSs controlled in a task-priority manner can perform automated
high resolution imagery and data collection, identifying and mitigating potential
damage and disruptions to marine ecosystems.

\emph{SDG 9 Industry, Innovation, and Infrastructure} aims to "Build resilient
infrastructure, promote inclusive and sustainable industrialization and foster
innovation. This can be achieved by enhancing underwater infrastructure reliability
through regular and automated inspection and maintenance, reducing the risk of
structural failures and environmental damage. This thesis also promotes innovation
in the field of maritime robotics, leading to more advanced technologies for
sustainable ocean management.

Finally, \emph{SDG 13 Climate Action} aims to "Take urgent action to combat climate
change and its impacts". UVMSs can facilitate oceanographic research by collecting
environmental data and monitoring climate change, laying the groundwork for actions
to combat the effects of climate change. UVMSs can support the development of
sustainable marine energy solutions, such as offshore wind farms and wave energy.

In summary, by aligning with key SDGs, this research demonstrates the potential
for sustainable innovation in underwater systems, contributing to a resilient 
and environmentally conscious future.

% ------------------------------------------------------------ Literature Review
\newpage
\section{Literature Review}

Some of the earliest works on task-priority control were introduced in 
\cite{hanafusa1981}. This study proposed a method for controlling redundant 
manipulators through a task-priority control scheme, enabling the simultaneous 
execution of two tasks. Specifically, the framework was applied to a 7-DOF 
manipulator tasked with tracking a desired end-effector position while 
maintaining a constant arm posture. This approach was later expanded in 
\cite{nakamura1987}, where a more comprehensive formulation was presented, 
along with the introduction of potential functions for obstacle avoidance.

Building on the foundations laid by \cite{hanafusa1981}, \cite{nakamura1987}, 
and others, the work in \cite{siciliano1991} extended the task-priority 
framework to accommodate an arbitrary number of tasks. The proposed formulation 
employs a recursive approach, enabling efficient computation of joint 
velocities. A simulation study demonstrated the method's effectiveness in 
controlling a 7-DOF planar manipulator performing three tasks, including 
obstacle avoidance. The challenges posed by algorithmic and kinematic 
singularities were later addressed in \cite{chiaverini1997}. While kinematic 
singularities can lead to excessively large joint velocities, algorithmic 
singularities may cause strict task priority to break down. To mitigate these 
issues, the study introduced damped pseudo-inverse matrices as a solution.

One notable limitation of the frameworks discussed so far is their assumption 
of decoupled kinematics and dynamics. However, in real-world 
systems—particularly in underwater vehicle-manipulator systems (UVMSs)—this 
assumption does not hold. To address this limitation, \cite{khatib1987} 
proposed a framework for redundancy resolution at the dynamic level. This 
approach incorporates task dynamics, employs feedback linearization techniques, 
and utilizes dynamically consistent generalized inverses \cite{khatib1995} to 
maintain strict task priority. However, this method relies heavily on an 
accurate dynamic model, which can be a tradeoff for addressing coupled dynamics 
and kinematics. The framework was further extended in \cite{khatib2004,sentis2004}, 
allowing it to handle an arbitrary number of tasks effectively. In this work, a 
high-DOF humanoid robot was successfully controlled in a simulation study.

Stability analysis of kinematic-level priority-based control schemes for 
redundant manipulators was conducted in \cite{antonelli2009}. Using a 
Lyapunov-based approach, the study provided sufficient conditions for stability 
based on control gains and task design. However, the proposed method assumes 
decoupled kinematics and dynamics, making it less directly applicable to UVMSs 
without additional assumptions regarding the relative speeds of robot dynamics 
and inverse kinematics.

A task-priority control scheme specifically applied to UVMSs was presented in 
\cite{antonelli1998}. This work employed a kinematic-level task-priority 
control approach to control a 6-DOF articulated underwater vehicle (AUV) 
equipped with a 3-DOF planar manipulator arm. While the simulation results were 
promising, the study did not account for the system's dynamics.

In recent years, approaches that incorporate dynamic considerations have been 
developed. For example, \cite{basso2020} introduced a method that uses control 
Lyapunov function-based quadratic programming to address control allocation, 
dynamic control, and redundancy resolution in redundant robotic systems. Given 
the challenges posed by disturbances and modeling inaccuracies in UVMS 
dynamics, \cite{iversflaten2022} proposed a dynamic controller based on sliding 
mode control. A simulation study demonstrated the effectiveness of this method 
in handling these challenges. Regarding physical experiments on UVMSs, there 
seem to be a lack of research that compares the performance of the proposed 
task-priority control methods in a real-world setting.



% Might want to include some articles on vehicle modeling?

% ----------------------------------------------------------------- Assumptions
\section{Assumptions}
\begin{enumerate}
    \item \emph{Assumption 1}
    \item \emph{Assumption 2}
    \item \emph{Assumption 3}
\end{enumerate}

% --------------------------------------------------------------- Contributions
\section{Contributions}

The main contributions as presented in this thesis are as follows:
\begin{itemize}
    \item \emph{A Python library for generating dynamic models of UVMSs, allowing
        for rapid prototyping of models.}
    \item \emph{A simulator in C++, together with a set of task-priority controllers
        ,using the generated models to simulate the behavior of UVMSs.}
    \item \emph{A simulation study comparing velocity and acceleration level
        task-priority control methods for light-UVMS.}
\end{itemize}

% -------------------------------------------------------------- Thesis Outline
\section{Thesis Outline}

\emph{Chapter 2} gives an overview of the background and preliminaries of the
thesis. The chapter introduces the mathematical tools and concepts used throughout
the thesis. \emph{Chapter 3} presents the mathematical modelling of underwater
vehicle manipulator systems (UVMs). \emph{Chapter 4} introduces the Python library
\pymuvs and explains how it can be used to model underwater vehicles. \emph{Chapter 5}
presents the simulation study comparing velocity and acceleration level task-priority
control methods for light-UVMS. \emph{Chapter 6} concludes the thesis and outlines
possible future work.
