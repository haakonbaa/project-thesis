\chapter{Background and Preliminaries}

% -----------------------------------------------------------------------------
\section{SO(3) and SE(3)}

We denote by $\SO$ the special orthogonal group in three dimensions, which is
the group of all $3 \times 3$ rotation matrices with determinant equal to one
\begin{align}
    \SO = \{ \bm{R} \in \mathbb{R}^{3 \times 3} \mid \bm{R}^T \bm{R} = \mathbb{I}, \det(\bm{R}) = 1 \}.
\end{align}
Intuitively, $R^T R = \mathbb{I}$ means that the columns of $R$ are orthonormal,
and $\det(R) = 1$ means that $R$ is a proper rotation, i.e., it does not include
a reflection. When specifying reference frames, we say that
\begin{align}
    \bm{p}^b = \bm{R}_{a}^b \bm{p}^a
\end{align}
meaning that the rotation matrix $\bm{R}_{a}^b$ rotates a vector from frame $a$
to frame $b$. Some important properties of $\bm{R}_a^b \in \SO$ are as follows \cite{modsim}:
\begin{itemize}
\item $\bm{R}_a^b = (\bm{R}_b^a)^{-1} = (\bm{R}_b^a)^T$
\item $\bm{R}_a^b \bm{R}_b^c = \bm{R}_a^c$
\end{itemize}
Rotations are often parameterized using Euler angles, quaternions, angle-axis or
rodriquez parameters. In this thesis, euler angles are used for simplicity. A
rotation matrix $\bm{R}$ can be represented by a rotation about the $x$-axis followed
by rotations about the $y$- and $z$-axes respectively. The rotation matrix is then \cite{modsim}
\begin{align}
    \bm{R} = \bm{R}_z(\psi) \bm{R}_y(\theta) \bm{R}_x(\phi),
\end{align}
where
\begin{subequations}
\begin{align}
    \bm{R}_x(\phi) &= \begin{pmatrix}
        1 & 0 & 0 \\
        0 & \cos(\phi) & -\sin(\phi) \\
        0 & \sin(\phi) & \cos(\phi)
    \end{pmatrix}, \\
    \bm{R}_y(\theta) &= \begin{pmatrix}
        \cos(\theta) & 0 & \sin(\theta) \\
        0 & 1 & 0 \\
        -\sin(\theta) & 0 & \cos(\theta)
    \end{pmatrix}, \\
    \bm{R}_z(\psi) &= \begin{pmatrix}
        \cos(\psi) & -\sin(\psi) & 0 \\
        \sin(\psi) & \cos(\psi) & 0 \\
        0 & 0 & 1
    \end{pmatrix}.
\end{align}
\end{subequations}
A disadvantage of using Euler angles is the well known problem of gimbal lock.
This occurs when two of the three axes align, causing the rotation matrix to lose
a degree of freedom. When this happens the mass matrix of the system becomes singular
and the system loses a degree of freedom. It is therefore impossible to determine
the derivative of the Euler angles.

{\color{red}
Write more about
\begin{itemize}
    \item Definition of angular velocity and skew-symmetric operators
    \item SE(3)
    \item Quaternions
    \item twist from euler angles and quaternions
\end{itemize}
}

% -----------------------------------------------------------------------------
\section{Pseudo-Inverse and Null Space Projections}
\label{sec:pseudoinverse}

For a given matrix $\bm{A} \in \mathbb{R}^{n\times m}$ we define the Moore-Penrose
pseudo-inverse $\bm{A}^{+} \in \mathbb{R}^{m\times n}$ as the unique matrix
satisfying the following four properties \cite{penrose1955}:
%\begin{enumerate}
%\item $\bm{A}\bm{A}^{+}\bm{A} = \bm{A}$
%\item $\bm{A}^{+}\bm{A}\bm{A}^{+} = \bm{A}^{+}$
%\item $(\bm{A}\bm{A}^{+})^T = \bm{A}\bm{A}^{+}$
%\item $(\bm{A}^{+}\bm{A})^T = \bm{A}^{+}\bm{A}$
%\end{enumerate}
\begin{subequations}
    \begin{align}
        &\textrm{1. } & \bm{A}\bm{A}^{+}\bm{A} &= \bm{A} \\
        &\textrm{2. } & \bm{A}^{+}\bm{A}\bm{A}^{+} &= \bm{A}^{+} \\
        &\textrm{3. } & (\bm{A}\bm{A}^{+})^T &= \bm{A}\bm{A}^{+} \\
        &\textrm{4. } & (\bm{A}^{+}\bm{A})^T &= \bm{A}^{+}\bm{A}
    \end{align}
\end{subequations}
Note that in the general case, the transpose operator can be replaced by the
conjugate transpose (hermitian) operator for matrices over the complex field $\mathbb{C}$ \cite{penrose1955}.
It can be shown that for a full-column rank matrix $\bm{A}$, the pseudo-inverse is
\begin{align}
    \bm{A}^{+} = (\bm{A}^T \bm{A})^{-1} \bm{A}^T
\end{align}
and for a full-row rank matrix $\bm{A}$, the pseudo-inverse is
\begin{align}
    \bm{A}^{+} = \bm{A}^T (\bm{A} \bm{A}^T)^{-1}
\end{align}
In the general case, weather or not $\bm{A}$ is full rank, the pseudo-inverse can allways
be computed using the singular value decomposition (SVD) of $\bm{A}$. The SVD of $\bm{A}$ is
\begin{subequations}
\begin{align}
    \bm{A} &= \bm{U} \bm{\Sigma} \bm{V}^T \\
    \bm{U} &\in \mathbb{R}^{n\times n} \textrm{ unitary} \\
    \bm{V} &\in \mathbb{R}^{m\times m} \textrm{ unitary} \\
    \bm{\Sigma} &= \operatorname{diag}(\sigma_1, \sigma_2, \ldots, \sigma_r, 0, \ldots, 0) \in \mathbb{R}^{n\times m} \\
    \sigma_1 &\geq \sigma_2 \geq \ldots \geq \sigma_r > 0 \\
    r &= \operatorname{rank}(\bm{A})
\end{align}
\end{subequations}
Where a matrix $U$ is unitry iff $U^T U = U U^T = \mathbb{I}$. The pseudo-inverse
of $\bm{A}$ can then be computed as
\begin{subequations}
\begin{align}
    \bm{A}^{+} &= \bm{V} \bm{\Sigma}^{+} \bm{U}^T \\
    \bm{\Sigma}^{+} &:= \operatorname{diag}(\sigma_1^{-1}, \sigma_2^{-1}, \ldots, \sigma_r^{-1}, 0, \ldots, 0)
\end{align}
\end{subequations}

The singular value
decompositoin will be important when dealing with singularity robust task priority
control later on in this thesis.

Consider the optimization problem
\begin{align}
    \min_{\bm{x} \in \mathbb{R}^m} || \bm{A} \bm{x} - \bm{b} ||
\end{align}
The solutoin to this problem is given by
\begin{align}
    \bm{x} = \bm{A}^{+} \bm{b} + (\mathbb{I} - \bm{A}^{+} \bm{A}) \bm{z}
\end{align}
where $\bm{z} \in \mathbb{R}^m$ is an arbitrary vector. The term $(\mathbb{I} - \bm{A}^{+} \bm{A}) \bm{z}$
is the null space projection of $\bm{z}$ onto the null space of $\bm{A}$. Parts
of this fact can bee seen by noting that
\begin{align}
    \bm{A}\left(\mathbb{I} - \bm{A}^{+} \bm{A}\right) = \bm{A} - \bm{A} = 0
\end{align}

% -----------------------------------------------------------------------------
\section{Lagrange's Equation of Motion}
Lagrange's equation of motion is a formulation of the equations of motion of a
system in terms of generalized coordinates. The equations are derived from the
d'Alembert principle. It can be shown that the equations of motion derived from
Lagrange's equation are equivalent to the equations of motion derived from
the classical Newton-Euler equations \cite{modsim}. There are however some advantages to using
Lagrange's equation. For example, the equations are easier to derive for complex
systems where the kinetic and potential energy is known. With modern computer
algebra systems (CAS), the equations can be derived automatically, which is
a great advantage for complex systems.

Consider a system with $n$ generalized coordinates $q_1, q_2, \ldots, q_n$, and
their time derivatives $\dot{q}_1, \dot{q}_2, \ldots, \dot{q}_n$. Collect the
generalized coordinates in the vector $\bm{q} \in \mathbb{R}^n$ and the time
derivatives in the vector $\bm{\dot{q}} \in \mathbb{R}^n$. The kinetic energy
of the system is then a function $T$ of $\bm{q}$ and $\bm{\dot{q}}$
\begin{align}
    T = T(\bm{q}, \bm{\dot{q}})
\end{align}
and the potential energy is a function $U$ of $\bm{q}$
\begin{align}
    U = U(\bm{q})
\end{align}
Define the lagrangian $L$ as the difference between the kinetic and potential
energy
\begin{align}
    L(\bm{q}, \bm{\dot{q}}) = T(\bm{q}, \bm{\dot{q}}) - U(\bm{q})
\end{align}
The equations of motion are then given by \cite{modsim}:
\begin{align}
    \frac{d}{dt} \left( \nabla_{\dot{\bm{q}}} L \right) - \nabla_{\bm{q}} L = \bm{\tau}
\end{align}
where the $i$-th element of the vector $\bm{\tau}$ is the generalized force acting
on the $i$-th generalized coordinate.
\begin{align}
    \tau_i = \sum_{k=1}^N \frac{\partial \bm{r}_k}{\partial q_i} \cdot \bm{F}_k
\end{align}
where $\bm{r}_k$ is the position of the $k$-th force and $\bm{F}_k$ is the force
acting on this position. In cases where the kinetic energy is on the form
\begin{align}
    T = \frac{1}{2} \bm{\dot{q}}^T \bm{J}^T(\bm{q}) \bar{\bm{M}} \bm{J}(\bm{q}) \bm{\dot{q}}
\end{align}
and the potential energy is a result of some potential field, the equations of
motion can be written as
\begin{align}
    \bm{M}(\bm{q}) \ddot{\bm{q}} + \bm{C}(\bm{q}, \dot{\bm{q}}) \dot{\bm{q}} + \bm{g}(\bm{q}) = \bm{\tau}
\end{align}
where $\bm{M}(\bm{q})$ is symmetric and positive definite, $\bm{g}(\bm{q})$ is the
gradient of the potential field.

% -----------------------------------------------------------------------------
\section{Notation}

The mathematical notation is primarily based on the notation used in \cite{modsim},
\cite{fossen2021}, and \cite{sola2017}. We denote the skew-symmetric operator of a vector $\bm{v} \in \mathbb{R}^3$ as
\begin{subequations}
\begin{align}
    [\cdot]_{\times} &: \mathbb{R}^3 \to \mathbb{R}^{3\times 3} \\
    [\bm{v}]_{\times} &= \begin{bmatrix}
        0 & -v_3 & v_2 \\
        v_3 & 0 & -v_1 \\
        -v_2 & v_1 & 0
    \end{bmatrix}
\end{align}
\end{subequations}
where for any two vectors $\bm{u}, \bm{v} \in \mathbb{R}^3$
\begin{align}
    \bm{u} \times \bm{v} = [\bm{u}]_{\times} \bm{v}
\end{align}
Adopting notation from \cite{fossen2021}, the pose of a 6-DOF rigid body
is denoted by the position $\bm{p} \in \mathbb{R}^3$ and the attitude, paramaterized
in euler angles as $\bm{\Theta} \in \mathbb{R}^3$. Note that the position is
desribed in North-East-Down (NED) coordinates. The pose of the body is then
\begin{align}
    \bm{\eta}^T = \begin{bmatrix} \bm{p}^T & \bm{\Theta}^T \end{bmatrix}^T
\end{align}
The generalized body-fixed velocities are given by
\begin{align}
    \bm{\nu}^T = \begin{bmatrix} \bm{v}^T & \bm{\omega}^T \end{bmatrix}^T
\end{align}
where $\bm{v} \in \mathbb{R}^3$ is the linear velocity and $\bm{\omega} \in \mathbb{R}^3$
is the angular velocity.
