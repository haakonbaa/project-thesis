\chapter{PyMUVs}
\emph{\textbf{Py}MUVs \textbf{M}odels \textbf{U}nderwater \textbf{V}ehicle\textbf{s}}

% -----------------------------------------------------------------------------
\section{Introduction}

\iffalse
\pymuvs{ }is a Python library for mathematically modelling underwater vehicles. The
library is designed to be modular and easy to use, and is built on top of NumPy
\cite{numpy} and SymPy \cite{sympy}. By defining the a set of links, as well as
their properties, such as mass, inertia, volume, linear drag, quadratic drag,
external forces, and transformations between links,
\pymuvs{ }can compute a symbolic representation of the system on the
form
\begin{align}
    \bm{M}(\bm{q}) \ddot{\bm{q}} + \bm{C}(\bm{q}, \dot{\bm{q}}) \dot{\bm{q}} +
    \bm{D}(\bm{q}, \dot{\bm{q}}) \dot{\bm{q}} + \bm{g}(\bm{q}) = \bm{J}^T(\bm{q}) \bm{B} \bm{u},
    \label{eq:pymuvs:eom}
\end{align}
where the matrices $\bm{M}$, $\bm{C}$, $\bm{D}$, $\bm{g}$, $\bm{J}$, and $\bm{B}$ are
computed symbolically and can be used to simulate the system. This is especially
useful when implementing model-based controllers that require the dynamics of the
system to be on the form of, or similar to \autoref{eq:pymuvs:eom}. The library
also supports exporting the symbolic representation, and the whole model to C++
code, which can be used in real-time simulations or for significantly faster
simulations.
The library, written in Python, allows for quick and easy prototyping of
models, with the flexibility to add or remove links and transformations.
The library is open-source and can be found at
\url{https://github.com/haakonbaa/pymuvs}. 
\fi

\pymuvs{} is a Python library designed for the mathematical modeling of 
underwater vehicles. Built on top of NumPy~\cite{numpy} and SymPy~\cite{sympy}, 
the library emphasizes modularity, usability, and flexibility. By defining a 
set of links along with their associated physical properties—such as mass, 
inertia, volume, linear drag, quadratic drag, external forces, and 
transformations between links—\pymuvs{} can compute a symbolic representation 
of the system dynamics in the form:

\begin{align}
    \bm{M}(\bm{q}) \ddot{\bm{q}} + \bm{C}(\bm{q}, \dot{\bm{q}}) \dot{\bm{q}} +
    \bm{D}(\bm{q}, \dot{\bm{q}}) \dot{\bm{q}} + \bm{g}(\bm{q}) = \bm{J}^T(\bm{q}) \bm{B} \bm{u},
    \label{eq:pymuvs:eom}
\end{align}

where the matrices $\bm{M}$, $\bm{C}$, $\bm{D}$, $\bm{g}$, $\bm{J}$, and 
$\bm{B}$ are computed symbolically and can be leveraged for system simulation. 
These matrices represent the mass, Coriolis, damping, gravitational and buoyancy
forces, Jacobian and input matrices, respectively.
This symbolic representation is particularly valuable when designing 
model-based controllers that rely on system dynamics expressed in or derived 
from \autoref{eq:pymuvs:eom}.

The library also supports exporting symbolic expressions and the entire system 
model to C++ code. This feature enables real-time simulations and significantly 
accelerates computational performance, which is essential for time-critical 
applications.

Written in Python, \pymuvs{} facilitates rapid prototyping and iterative 
development, allowing users to easily add or remove links and transformations. 
The library is open-source and publicly available at 
\url{https://github.com/haakonbaa/pymuvs}.



% -----------------------------------------------------------------------------
\subsection{Motivation and Importance}

The primary motivation behind the development of \pymuvs{} was to create a 
robust tool for modeling the dynamics of underwater vehicles. Currently, there 
is a notable lack of software that allows users to intuitively define the 
parameters of an underwater vehicle and directly obtain the model matrices 
required for model-based control. While open-source simulators, such as Gazebo, 
streamline the mathematical modeling and simulation of these systems, they do 
not explicitly provide the system matrices. This limitation arises because such 
simulators fundamentally do not rely on system matrices for simulation 
purposes. \pymuvs{} addresses this gap by directly computing the model matrices 
from a general parameterization of underwater vehicle-manipulator systems (UVMSs).  

One key advantage of generating these matrices symbolically is their 
portability across programming languages. SymPy, the symbolic mathematics 
library used by \pymuvs{}, includes built-in functionality for exporting 
symbolic expressions to languages such as C, Fortran, and MATLAB/Octave. This 
greatly reduces the effort required to produce efficient C/C++ code for 
evaluating these matrices, a process demonstrated successfully in this project.  

As previously mentioned, physical experiments are planned with the Eelume 500 
robot. Since the interface for this UVMS is implemented in C++, it is a natural 
choice to also implement the control algorithms in the same programming 
language. Furthermore, task-priority controllers are computationally intensive, 
requiring frequent matrix inversions, null-space projections, and Jacobian 
calculations. Using an efficient, compiled language such as C++ offers 
significant advantages, including low-level optimization capabilities, static 
typing, and access to an extensive set of libraries. These factors strongly 
justify the need for a tool capable of generating model matrices directly in 
C++.  

In addition to model matrices, the \pymuvs{} library also generates the 
Jacobians required for task-priority control algorithms. Jacobians represent 
derivatives of transformation functions, necessitating some form of symbolic 
differentiation. \pymuvs{} seamlessly integrates this functionality, enabling 
the computation of transformation matrices and Jacobians directly from a 
defined model. This feature significantly reduces the complexity of both 
defining and evaluating Jacobians in C++. 

% -----------------------------------------------------------------------------
\section{Architecture and Design}

The \pymuvs{} library is structured into four submodules in addition to the 
primary package, referred to as \emph{pymuvs}. These four submodules are as follows:

\begin{itemize}
    \item \emph{se3}: This submodule contains an implementation of a symbolic $\SE$ 
        class, which is extensively utilized in the core \emph{pymuvs} package.
        This submodule relies heavily on the mathematics presented in
        \autoref{sec:bp:so3_se3} and \autoref{sec:bp:se3}.

    
    \item \emph{util}: This submodule provides utility functions for matrix 
        manipulation, such as the skew-operator, time differentiation, and derivatives 
        of matrices and vectors.
    
    \item \emph{tp}: This submodule defines classes for tasks $\bm{\sigma}$ and 
        desired tasks $\bm{\sigma}_d$.
    
    \item \emph{codegen}: This submodule includes tools for generating C++ source 
        and header files from dynamic models, as well as task Jacobians and their derivatives.
\end{itemize}

The \pymuvs{} package serves as the central component of the library, and uses
much of the mathematics presented in \autoref{ch:modeling} and \autoref{sec:lagrange}.
The  subsequent sections will provide an overview of each submodule, along with 
brief examples demonstrating fundamental use cases. It is important to note 
that this overview does not encompass all functionalities of the library but 
aims to highlight the most essential features of each submodule.



% ------------------------------------------------------------------------- se3
\subsection{The \texttt{se3} Submodule}

The \texttt{se3} submodule provides functionality for working with the special Euclidean group $SE(3)$. Built on top of \texttt{SymPy}, it offers tools for generating elements of $SE(3)$ from symbolic expressions, as well as operations for computing the inverse and performing multiplication of these elements.

A simple example demonstrating the creation of a rotation matrix from Euler angles is shown below:

\begin{lstlisting}[style=custompython,
    caption={Creating a rotation matrix from Euler angles.},
    label={lst:usage:se30}]
import sympy as sp
from pymuvs.se3 import rot_x, rot_y, rot_z

phi, theta, psi = sp.symbols('ϕ θ ψ')
T = rot_z(psi) @ rot_y(theta) @ rot_x(phi)
\end{lstlisting}

The resulting symbolic output of this example is shown here:

\begin{lstlisting}[style=customtxt,
    caption={Output of \autoref{lst:codegen}.},
    label={lst:usage:se3}]
SE3(rotation=Matrix([
    [cos(θ)*cos(ψ), sin(θ)*sin(ϕ)*cos(ψ) - sin(ψ)*cos(ϕ), sin(θ)*cos(ψ)*cos(ϕ) + sin(ψ)*sin(ϕ)],
    [sin(ψ)*cos(θ), sin(θ)*sin(ψ)*sin(ϕ) + cos(ψ)*cos(ϕ), sin(θ)*sin(ψ)*cos(ϕ) - sin(ϕ)*cos(ψ)],
    [ -sin(θ), sin(ϕ)*cos(θ), cos(θ)*cos(ϕ)]]),
translation=Matrix([[0], [0], [0]])
\end{lstlisting}

These examples illustrate how to create a variable $T \in \SO$ representing a pure rotation using Euler angles. The resulting transformations can be composed to construct arbitrary transformations. This group representation is extensively used in other modules of the library to define and manipulate transformations between links in a robotic model.



% --------------------------------------------------------------------- util
\subsection{The \texttt{util} Submodule}

The \emph{util} submodule provides utility functions for matrix operations, including tools for time differentiation and Jacobian computation. One notable function is \texttt{skew}, which calculates the skew-symmetric matrix of a vector. Additionally, the module includes functions such as \texttt{time\_diff\_matrix} for computing the time derivative of matrices and \texttt{jacobian} for differentiating vectors with respect to other vectors.

An example demonstrating the time differentiation of a matrix is shown below:

\begin{lstlisting}[style=custompython,
    caption={Time differentiation of a matrix.},
    label={lst:usage:util:time_diff_matrix}]
import sympy as sp
from pymuvs.util import time_diff_matrix

x, y = sp.symbols('x y')
dx, dy = sp.symbols('dx dy')

A = sp.Matrix([[x, x**2], [y, y**2]])
dA = time_diff_matrix(A, [x, y], [dx, dy])
print(dA)
\end{lstlisting}
The output of this operation is as follows:
\begin{lstlisting}[style=customtxt,
    caption={Output of \autoref{lst:usage:util:time_diff_matrix}.},
    label={lst:usage:util:time_diff_matrix:out}]
Matrix([[dx, 2*dx*x], [dy, 2*dy*y]])
\end{lstlisting}

These examples illustrate how to compute the time derivative of a matrix, an operation frequently used in the core of the \emph{pymuvs} package.

Another example demonstrates how to compute the Jacobian of a vector with respect to another vector:

\begin{lstlisting}[style=custompython,
    caption={Computing the Jacobian of a vector.},
    label={lst:usage:util:jacobian}]
import sympy as sp
from pymuvs.util import jacobian

x, y, z = sp.symbols('x y z')
q = sp.Matrix([x, y, z])

v = sp.Matrix([[x + y**2 + x*z], [x*y*z]])
dv = jacobian(v, q)
print(dv)
\end{lstlisting}

The resulting output is as follows:

\begin{lstlisting}[style=customtxt,
    caption={Output of \autoref{lst:usage:util:jacobian}.},
    label={lst:usage:util:jacobian:out}]
Matrix([[z + 1, 2*y, x], [y*z, x*z, x*y]])
\end{lstlisting}


These examples highlight the functionality of the \texttt{jacobian} function, which is widely utilized in the library. It is worth noting that the implementations of these functions are concise, as the heavy computational work is delegated to \texttt{SymPy}. The functions are designed to handle common input types gracefully and include robust error checking to ensure reliability.


% --------------------------------------------------------------------- tp
\subsection{The \texttt{tp} submodule}
The \emph{tp} submodule contains the two classes \texttt{Task} and \texttt{DesiredTask}
which are used to represent tasks and desired tasks in a task-priority controller.
The \texttt{Task} class contains symbolic representations of the familiar $\bm{f}(\bm{q})$,
$\bm{J}(\bm{q})$ and $\dot{\bm{J}}(\bm{q},\dot{\bm{q}})$ matrices, while the \texttt{DesiredTask} class contains
$\bm{\sigma}_d(t)$, $\dot{\bm{\sigma}}_d(t)$ and $\ddot{\bm{\sigma}}_d(t)$. 
These classes are used in the \emph{codegen} submodule to generate C++ code for
the task Jacobians and their derivatives. An example of how to use these classes
is shown below.

\begin{lstlisting}[style=custompython,
    caption={Creating a task from a function.},
    label={lst:usage:tp:task}]
import sympy as sp
from pymuvs.tp import fn_to_task

x, y = sp.symbols('x y')
dx, dy = sp.symbols('dx dy')
f = sp.Matrix([x**2*y**2, x**2+y**2])
task = fn_to_task(f, [x, y], [dx, dy])
print(task.f)
print(task.J)
print(task.dJ)
\end{lstlisting}
\begin{lstlisting}[style=customtxt,
    caption={Output of \autoref{lst:usage:tp:task}.},
    label={lst:usage:tp:task:out}]
Matrix([[x**2*y**2], [x**2 + y**2]])
Matrix([[2*x*y**2, 2*x**2*y], [2*x, 2*y]])
Matrix([[2*dx*y**2 + 4*dy*x*y, 4*dx*x*y + 2*dy*x**2], [2*dx, 2*dy]])
\end{lstlisting}
\autoref{lst:usage:tp:task} and \autoref{lst:usage:tp:task:out} shows how to create a task from a symbolic
matrix. We can see that the task object contains the symbolic representation of
the task, as well as the Jacobian and its time derivative. This is very useful
when generating C++ code for a task-priority controller.

\begin{lstlisting}[style=custompython,
    caption={Generating a desired task from a function.},
    label={lst:usage:tp:task_des}]
import sympy as sp
from pymuvs.tp import fn_to_task_desired

t = sp.symbols('t')
f = sp.Matrix([sp.sin(t), sp.cos(t)])
sigma_d = fn_to_task_desired(f, t)
print(sigma_d.sigma)
print(sigma_d.dsigma)
print(sigma_d.ddsigma)
\end{lstlisting}
\begin{lstlisting}[style=customtxt,
    caption={Output of \autoref{lst:usage:tp:task_des}.},
    label={lst:usage:tp:task_des:out}]
Matrix([[sin(t)], [cos(t)]])
Matrix([[cos(t)], [-sin(t)]])
Matrix([[-sin(t)], [-cos(t)]])
\end{lstlisting}
\autoref{lst:usage:tp:task_des} and \autoref{lst:usage:tp:task_des:out} shows
how to create a desired task from a symbolic matrix. We observe that the desired
task object contains the symbolic representation of the desired task, as well as
its first and second time derivative.





% --------------------------------------------------------------------- codegen
\subsection{The codegen submodule}
The \texttt{codegen} module is used to generate C++ code from a \texttt{Model}. The module
uses the \texttt{sympy.ccode} function to generate C code from
each expression in every matrix and combines them into a header and a source
file. The \texttt{Eigen} library \cite{eigen3} is used to represent the matrices.
This library is chosen for its performance, ease of use and wide adoption in the
robotics community. The library is header-only, meaning that it is easy to
include in other projects.


\begin{lstlisting}[style=customcpp,
    caption={Automatically generated C++ code from a model.},
    label={lst:codegen}]
#pragma once
#include <Eigen/Dense>
#include <math.h>
#include <stdexcept>

/*
    A mathematical model of a robot on the form
        M(q) * ddq + C(q, dq) dq + D(q, dq) dq + g(q) = Jf(q) B u(z)
    z is a specified vector of inputs to the robot.

    M, C and D are n x n matrices
    J  is a  Nj x Nn  matrix
    Jf is a  Nn x Nb  matrix
    B  is a  Nb x Nm  matrix
    u  is a  Nm x  1  vector

    The values of nb, nj and m might be hard to calculate before the model is
    created.
    
 q = [xn, yn, zn, ϕ, θ, ψ, a1, a2, a3, a4]
dq = [dxn, dyn, dzn, dϕ, dθ, dψ, da1, da2, da3, da4]
 z = [τ1, τ2, τ3, τ4, f1s, f1p, f1b, f1t, f2s, f2p, f2b, f2t]
 u = [f1p, f1s, f1b, f1t, f2p, f2s, f2b, f2t, τ1, τ2, τ3, τ4]
*/

namespace Model {

constexpr int Nn  = 10;
constexpr int Nj  = 18;
constexpr int Nb  = 22;
constexpr int Nm  = 12;
constexpr int Nz  = 12; // number of inputs
constexpr int Nt  = 3;  // number of transforms

Eigen::MatrixXd M(Eigen::VectorXd q);
Eigen::MatrixXd C(Eigen::VectorXd q, Eigen::VectorXd dq);
Eigen::MatrixXd D(Eigen::VectorXd q, Eigen::VectorXd dq);
Eigen::VectorXd g(Eigen::VectorXd q);
Eigen::MatrixXd J(Eigen::VectorXd q);
Eigen::MatrixXd Jf(Eigen::VectorXd q);
Eigen::MatrixXd B();
Eigen::VectorXd u(Eigen::VectorXd z);
Eigen::MatrixXd Ju(Eigen::VectorXd z);
Eigen::VectorXd T0_to_b(Eigen::VectorXd q, Eigen::VectorXd x_0);
Eigen::VectorXd T1_to_b(Eigen::VectorXd q, Eigen::VectorXd x_1);
Eigen::VectorXd T2_to_b(Eigen::VectorXd q, Eigen::VectorXd x_2);
Eigen::VectorXd ddq(Eigen::VectorXd q, Eigen::VectorXd dq, Eigen::VectorXd z);
Eigen::VectorXd u_to_z(Eigen::VectorXd u);


} // namespace Model
\end{lstlisting}

\autoref{lst:codegen} shows the header file generated from a model. The output
file contains functions for computing the various matrices in the model, as well
as transformations between the reference frame of the links to the base frame,
begin the functions named \texttt{T0\_to\_b}, \texttt{T1\_to\_b} and \texttt{T2\_to\_b}.
The header file also contains contains a small description of the dynamic model,
as well as some of the \texttt{constexpr} values denoting the dimensions of the
matrices in the model. Using implicit sizes for the matrices allows for easier
development of the code using the model, as it does not need to be templated on
the size of the matrices if the model is changes. This does however come at the
cost of some debugging capabilities, as the compiler will not catch dimension
mismatches between matrices at compile time. This was a trade-off made to make
the code using the model more readable and easier to develop.

% -----------------------------------------------------------------------------
\section{Usage and Examples}

The core of the \pymuvs{} library are the classes \texttt{Link}, \texttt{Robot}, \texttt{Model},
and \texttt{Wrench}. These classes are used to define the physical properties of the
robot, as well as the transformations between the links. The \texttt{Link} class
represents a single link in the robot, and its properties are mass, inertia matrix,
volume, added mass matrix, linear damping matrix, quadratic damping matrix, center of
mass, center of buoyancy and a set of wrenches acting on the link. All of these
properties are defined numeric values, exept for the wrenches being its own class.
Consisting of a set of links, and the transformations
between them, the \texttt{Robot} class is used to define the UVMS. The \texttt{Robot} class
has the \texttt{get\_model} method, which returns a \texttt{Model} object representing the
dynamics of the robot. This function is arugably the most important and computationally
demanding function in the library, as it computes the symbolic representation of the
robot dynamics.

\subsection{Model Definition}

To demonstrate the usage of the \pymuvs{} library, a simple example is provided.
Consider a simple pendulum with a mass $m$ and length $l$. Assume that we are in
a East-North-Up coordinate system and that the pendulum is free to move in the
North-Up plane. The pendulum is affected by a gravitational force $g$. Putting
it all together, the model of the pendulum can be defined in \pymuvs{} as follows.

\begin{lstlisting}[style=custompython,
    caption={Dynamic model of a simple pendulum in \pymuvs{}.},
    label={lst:usage:pendulum}]
import sympy as sp
import numpy as np

from pymuvs import Link, Robot
from pymuvs.se3 import rot_x, trans

l = 1    # length of pendulum
m = 1    # mass of pendulum
g = 9.81 # gravity

theta, dtheta = sp.symbols('θ dθ')
# Transform from link- to inertial-frame
Tbn = rot_x(theta) @ trans(0, 0, -l)
end_mass = Link(m, 0, np.zeros((3, 3)),
        np.zeros((6, 6)), np.zeros((6, 6)),
        np.zeros((6, 6)))

# Robot has links, each of which has their own transform
pendulum = Robot(links=[end_mass],
    transforms=[Tbn],
    params=[theta],
    diff_params=[dtheta])

# gravity in -z direction
model = pendulum.get_model(gvec=np.array([0, 0, -g]))
print(f"{model.M = }")
print(f"{model.C = }")
print(f"{model.D = }")
print(f"{model.g = }")
\end{lstlisting}
\begin{lstlisting}[style=customtxt,
    caption={Output of \autoref{lst:usage:pendulum}.},
    label={lst:usage:pendulum:out}]
model.M = Matrix([[1.0]])
model.C = Matrix([[0]])
model.D = Matrix([[0]])
model.g = Matrix([[9.81*sin(θ)]])
\end{lstlisting}
\autoref{lst:usage:pendulum} and \autoref{lst:usage:pendulum:out} shows how to define a simple pendulum
in \pymuvs{}. The model is defined by a single link, representing the mass at the
end of the pendulum. The transformation from the link frame to the inertial frame
is defined by the rotation matrix \texttt{Tbn}. The \texttt{Robot} object is then
created with the link and the transformation, and the model is computed using the
\texttt{get\_model} method. The output shows the mass matrix, Coriolis matrix, damping
matrix and the gravitational force acting on the pendulum. We recognize this
as the dynamics of a simple pendulum.
\begin{align}
    \ddot{\theta} + \frac{g}{l} \sin(\theta) = 0
\end{align}
Note that this simple example does not include any damping, buoyancy or added mass,
but it serves as a good starting point for understanding how to define a model in
\pymuvs{}.

\subsection{Validation and Testing}

Along with the development of the \pymuvs{} library, a comprehensive test suite
was created to ensure the correctness of the library. The test suit consists of
tests of functionality for each submodule. The genrated C++ code has been tested
manually, but as of now there are no automated tests for the generated code. The
unittests can be run by using the build-in \texttt{unittest} module in Python.
\begin{verbatim}
    python -m unittest discover -s test
\end{verbatim}
in the root of the repository. This will run all the tests in the \texttt{test}
directory. The tests are not 100\% coverage, but they cover the most important
functionality of the library. The tests also serve as examples of how to use the
library, and can be used as a reference when developing models using \pymuvs{}.
Examples of code generation and model definition can be found in the tests. Some
of the tests also compare the generated Jacobians with the analytical Jacobians
from \cite{fossen2021}.


% -----------------------------------------------------------------------------
\section{Installation}

Installing the \pymuvs{ }library is straightforward. The library used Git for
version control, and the code is hosted on \url{https://github.com/haakonbaa/pymuvs}.
To install the library and its dependencies, the following steps can be taken. It
is assumed that the user has Git installed on their system.
\begin{enumerate}
    \item Clone the repository using Git. This can be done by running the following
    command in a terminal.
    \begin{verbatim}
    git clone https://github.com/haakonbaa/pymuvs.git
    \end{verbatim}
    \item Install the dependencies. Optionally, a virtual environment can be created
    to avoid conflicts with other Python packages. This can be done by running:
    \begin{verbatim}
    python -m pip install -r requirements.txt
    \end{verbatim}
\item The library is structured as a Python package, and can be installed system-wide
    by running the following command in the root of the repository.
    \begin{verbatim}
    python -m pip install -e .
    \end{verbatim}
\end{enumerate}
The library is now installed and can be imported in Python by running
\begin{lstlisting}[caption={Importing the \pymuvs{ }package},
    style=custompython,
    label={lst:pymuvs:example}]
import pymuvs
\end{lstlisting}

% -----------------------------------------------------------------------------
\section{Simulator}
In addition to the \pymuvs{} library, a dedicated simulator was developed to 
model the dynamics of various underwater vehicles. The simulation results 
presented in this thesis are derived from models created using \pymuvs{}, with 
corresponding C++ code generated from these models. A custom C++ implementation 
was developed to utilize the \pymuvs{}-generated code for simulating the 
system. Throughout this thesis, this implementation is referred to as 
\textit{the simulator}.

The simulator employs the fourth-order fixed-step Runge-Kutta method, as 
described in \autoref{sec:runge-kutta}, to integrate the system dynamics. This 
method was selected due to its simplicity and reliable performance. 
Implementing the simulator in C++ offers several advantages. Firstly, C++ 
provides a highly efficient simulation environment, which is particularly 
valuable during controller design and parameter tuning. Additionally, 
controllers can be implemented directly in C++ and executed within the same 
simulation environment, an essential feature for real-time controller development.

The simulator also facilitates seamless integration with higher-level 
programming languages such as MATLAB or Python. Through C++ wrappers, 
controllers implemented in these languages can be executed alongside the 
simulation. Similarly, low-level programming languages like Rust, Go, or C can 
be compiled into shared libraries and dynamically linked with the C++ 
simulator, enabling flexible integration across different software ecosystems.

One significant benefit of using C++ is the ability to perform simulations 
faster than real-time. To demonstrate the simulator's performance, a benchmark 
test was conducted with the following parameters: a simulation time of $100 \, 
\mathrm{s}$, a time step of $\Delta t = 0.01$, and a simple PD-controller. The 
simulator successfully simulated a 3-link system with 10 states in an average 
time of $0.3757 \, \mathrm{s}$, with a standard deviation of $3.4908 \, 
\mathrm{ms}$, across 10 runs. This corresponds to a speedup factor of 
approximately $266$ times faster than real-time.

It is important to note that the mass matrix in \autoref{eq:pymuvs:eom} must be 
inverted four times per time step when evaluating $\ddot{\bm{q}}$. While the 
simulation's performance may vary based on factors such as controller 
complexity, model intricacy, and computational hardware, it is evident that the 
simulator offers a highly efficient and robust environment for dynamic system 
simulations.

% -----------------------------------------------------------------------------
\section{Conclusion}

The \pymuvs{} library provides a robust and modular framework for modeling 
underwater vehicle dynamics using symbolic mathematics. Built on top of NumPy 
and SymPy, it allows users to define system properties, compute symbolic 
representations of dynamic models, and export them efficiently to C++ for 
real-time applications. Its structure, divided into well-organized submodules, 
supports essential functionalities such as handling transformations, performing 
time differentiation, and generating Jacobians, ensuring flexibility and 
clarity in implementation.

One of the library's key strengths lies in its ability to bridge the gap 
between symbolic modeling and computational efficiency. By exporting symbolic 
expressions to optimized C++ code, \pymuvs{} enables faster simulations and 
supports computationally intensive applications like model-based controllers. 
This feature is particularly beneficial for real-time scenarios where 
performance and accuracy are critical. The library not only simplifies the 
process of modeling complex underwater systems but also provides a foundation 
for developing advanced control algorithms and simulation 
environments.
