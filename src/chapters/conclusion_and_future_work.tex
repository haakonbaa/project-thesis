\chapter{Conclusion and Future Work}

% ----------------------------------------------------------------- Future Work
\section{Future Work}

While the simulation and analysis presented in this thesis provide valuable 
insights into task-priority control for light-UVMSs, there are several topics
that could be explored further:

\begin{itemize}
    \item \textbf{Experimental Validation on Physical Systems}: A critical next step is to 
validate the simulation results on NTNU's Eelume 500 robot in real-world 
underwater conditions. This would account for additional complexities such as 
sensor noise, unmodeled hydrodynamic effects, and actuator 
limitations.

    \item \textbf{Integration of Thruster and Motor Dynamics}: Current simulations 
        assume ideal thruster and motor responses. Future work should incorporate 
        dynamic models of thrusters and motors, including time delays and saturation 
        effects, to improve simulation realism.

    \item \textbf{Simulation with Model Errors}: Investigating the performance of the 
        task-priority controller under model uncertainties, such as inaccurate 
        hydrodynamic parameters or thruster models, would provide valuable insights 
        into the controller's robustness.

    \item \textbf{Quaternion-based Orientation Control}: Euler angles are prone to gimbal lock, which can limit the controller's robustness in certain configurations. Future work could explore using quaternions for attitude control to eliminate this limitation and ensure continuous rotation representation.

    \item \textbf{Advanced Control Strategies}: Implementing more advanced control 
        strategies, such as adaptive control or sliding mode control, could improve
        the controller's performance in the presence of disturbances and model
        uncertainties.

\end{itemize}

By addressing these areas, future research can significantly bridge the gap between simulation results and real-world applications, advancing the deployment of light-UVMSs in practical underwater missions.




% ------------------------------------------------------------------ Conclusion
\section{Conclusion}

This thesis explored task-priority control methodologies for lightweight 
underwater vehicle-manipulator systems (light-UVMSs), specifically focusing on 
both kinematic- and dynamic-level approaches. A custom simulation framework was 
developed to compare the performance of these controllers in scenarios 
involving multi-task prioritization. Key findings from this work include:
\begin{itemize}
    \item \textbf{Dynamic-Level vs. Kinematic-Level Control}: Dynamic-level control 
        demonstrated superior task tracking and prioritization capabilities, 
        particularly in scenarios with conflicting task requirements. However, it comes 
        with higher computational demands compared to kinematic-level control.
    \item \textbf{Controller Robustness}: The use of variable damped least squares (VDLS) pseudoinverse significantly improved controller robustness near singular configurations, ensuring more stable force and torque outputs.
    \item \textbf{Simulation Framework}: A reusable simulation framework capable of integrating advanced control algorithms was successfully implemented. This provides a foundation for future research and experimental validation.
\end{itemize}

The results indicate that dynamic-level control is a promising approach for 
light-UVMS applications, offering better performance in task execution and 
prioritization. However, real-world validation remains necessary to confirm 
these findings under non-ideal environmental conditions.

The contributions of this thesis lay a foundation for future experimental 
efforts and lays a foundation for deploying advanced task-priority control 
strategies on real underwater robotic platforms. Continued research in this 
area holds the potential to significantly enhance the capabilities of 
light-UVMSs in environmental monitoring, underwater infrastructure inspection,
and intervention tasks.
