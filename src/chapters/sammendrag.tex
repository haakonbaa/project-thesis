Lette undervanns-farkostmanipulatorsystem, inkludert fjernstyrt undervanns
farkoster og artikulerte intervensjons-AUV-er, får økende betydning i undervannsindustrien som havneovervåking og miljøovervåking. Disse systemene representerer unike kontrollutfordringer på grunn av den sterke koblingen mellom bevegelser i basen on mainipulatorarmine, et fenomen som er mindre uttalt i tyngere arbeidsklasse-ROV-er. Denne avhandlingen utforsker prioritert oppgavestyring for å håndtere disse utfordringene på både kinematisk og dynamisk nivå. Et omfattende simuleringsrammeverk er utviklet for å muliggjøre implementering og sammenligning av disse kontrollmetodene, og dermed minimere avstanden mellom simulering og ytelse i den virkelige verden.
Kinematisk prioritert oppgavestyring fokuserer på reduntantløsninger for hastinghet og akselerasjon, mens dynamisk kontroll inkluderer tilbakekoblingslinearisering og dynamisk konsistente inverser for å sikre robust oppgaveutførelse til tross for modellfeil og forstyrrelser. Simuleringsresultatene fremhever styrkene og avveiningene ved begge kontrolltilnærmingene, og gir innsikt i deres ytelse, beregningsmessige effektivitet og robusthet under varierende operasjonelle forhold. Dette arbeidet legger grunnlaget for fremtidig eksperimentell validering på NTNU's Eelume 500 artikulerte intervensjons-AUV, og fremmer utvikling innen undervannsrobotikk.
