The project considers nonlinear control methods for lightweight underwater vehicle-manipulator
systems (light-UVMSs) including both remotely operated vehicles (ROVs) with a lightweight
base, and articulated intervention-AUVs (AIAUVs) (e.g. underwater snake robots) which are
lightweight and with no separate base. Lightweight UVMSs are increasingly taken into use in
several industries, including harbor control and environmental monitoring.

Light-UVMSs reveal a control challenge which has so far has been neglected: to efficiently
control a light-UVMS one should consider the coupling effects between its base and arm
motions. When the arm moves, this will make the base move as well, and vice versa. This
coupling is not so clear for large work-class ROVs where the motion of the heavy base can be
assumed almost decoupled from the motion of the arm. The coupling effects appear partly due
to both reaction forces and hydrostatic forces.

Since UVMSs possess both thrusters and a manipulator arm, these vehicles are considered
kinematically redundant. This makes them very suitable for task-priority control methods. These
methods can be divided into two main groups: velocity- and acceleration-level task-priority
control. Velocity-level control does not explicitly consider all the coupling effects between the
manipulator and the base motions, whereas acceleration-level control does. However, one must
have a sufficiently accurate mathematical model of the UVMS to correctly apply acceleration-
level control. Currently there is a lack of research that compares these methods for underwater
vehicles, especially related to experimental validation.

The task of this assignment is to compare and implement several task-priority control
approaches for light-UVMS and compare their behaviour both theoretically and in simulation.
The main focus should be a comparison between control approaches applied at the velocity and
acceleration levels. The simulation study should be designed to facilitate physical experiments
to validate the findings. NTNU has, through
the AURLab, several vehicles (Eelume, BlueROV, etc.) fit for such experiments.
\newpage
\begin{enumerate}
\item Do a literature survey on task-priority control for underwater vehicle manipulator
systems (UVMS), describe the state-of-the-art, with a focus on works with experimental
results.
\item Write a theoretical comparison of various task-priority methods and their application to
UVMS.
\item Do a simulation study comparing velocity and acceleration level task-priority control
methods for light-UVMS.
\end{enumerate}
The report shall be written in English and edited as a research report including Abstract,
Introduction with motivation, literature survey, contributions of the project work, and the outline
of the report. This is followed by the chapters describing the results of the project work,
simulation results and corresponding discussion, and a conclusion including a proposal for
further work. Source code should be provided in a .zip-file uploaded as an attachment, and with
with code listing enclosed in an appendix in the report. It is supposed that Department of
Engineering Cybernetics, NTNU, can use the results freely in its research work, unless otherwise
agreed upon, by referring to the student’s work.

{
    \color{red}
    Note
    \begin{itemize}
       \item remove the last paragraph?
    \end{itemize}
}
